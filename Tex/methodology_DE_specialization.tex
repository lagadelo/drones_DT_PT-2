% Specialized methodology details (Methodology D/E): design-space exploration and validation.
\subsection{Fulfilling $R_{DT}^\circ$1: Domain Space Exploration}

We operationalize $R_{DT}^\circ$1 by exploring a small, reviewer-auditable design space built around a single baseline controller (Alg.~\ref{alg:baseline}) and two parameter-only variation points. This keeps the controller structure identical across candidates (same sensing, same control loop, same safety shaping) and isolates the effect of changing gains/schedules.

\noindent\textbf{Variation points (explicit replacements).}
\begin{itemize}
    \item \textbf{VP-A (gain scheduling, Alg.~\ref{alg:baseline} line~\ref{algline:spacing}):} replace the baseline spacing update by an error-dependent gain.
    \begin{itemize}
        \item \emph{Baseline (line~\ref{algline:spacing}):}
        $v_i\leftarrow V + k_f(d_f-d^{\star}) - k_b(d_b-d^{\star})$
        \item \emph{Replacement (VP-A):}
        $v_i\leftarrow V + k_f\,g\big(|d_f-d^{\star}|\big)(d_f-d^{\star}) - k_b(d_b-d^{\star})$,
        where $g(\cdot)$ is a nondecreasing schedule with $g(0)=1$.
    \end{itemize}

    \item \textbf{VP-B (recovery gain scheduling, Alg.~\ref{alg:baseline} lines~\ref{algline:rec}--\ref{algline:cap}):} keep the same control structure but switch gains when $rec=1$.
    \begin{itemize}
        \item \emph{Baseline:} use $(k_f,k_b)$ always.
        \item \emph{Replacement (VP-B):} use $(k_f,k_b)\leftarrow (k_f^{\text{rec}},k_b^{\text{rec}})$ while $rec=1$ and revert to nominal gains when $rec=0$.
    \end{itemize}

    \item \textbf{VP-C (adaptive sensing, neighbor state awareness, Alg.~\ref{alg:baseline} line~\ref{algline:spacing}):} disable back-pressure ($k_b$ term) during spare integration to prevent speed collapse.
    \begin{itemize}
        \item \emph{Baseline:}
        $v_i\leftarrow V + k_f\,g\big(|d_f-d^{\star}|\big)(d_f-d^{\star}) - k_b(d_b-d^{\star})$
        \item \emph{Rationale:} When a drone detects that its successor is in INCOMING mode (spare joining), it disables the back-pressure term to allow predecessor to maintain cruise speed rather than decelerating in response to the short transient gap.
        \item \emph{Replacement (VP-C):}
        $v_i\leftarrow V + k_f\,g\big(|d_f-d^{\star}|\big)(d_f-d^{\star}) - k_b(d_b-d^{\star})\cdot\mathbb{1}_{[\text{succ\_not\_incoming}]}$,
        where $\mathbb{1}_{[\text{succ\_not\_incoming}]} = 0$ if successor is in INCOMING mode (detected via local broadcast), else $1$.
    \end{itemize}
\end{itemize}

\subsection{Spare Insertion Stabilization: Three-Phase Control with Distance-Weighted Sensing}

While VP-C prevents the predecessor from crashing during spare insertion, the spare itself requires bounded acceleration to avoid overshoot and extended settling. We address this through a three-phase control scheme where the spare's velocity command is shaped based on its temporal position relative to insertion and its spatial position relative to neighbors.

\subsubsection{Control Phases}

\noindent\textbf{Phase 1: Soft Entry ($0 \leq t < 0.5$\,s).} The spare's initial speed is set to $0.6 \cdot V$ (60\% of fleet nominal speed) to create a smooth relative velocity profile. During this phase, the velocity ramps linearly toward nominal while receiving back-regulation from the rear gap:
\begin{equation}
v_{\text{spare}}^{(1)} = V_{\text{entry}} + \frac{t}{t_1} (V - V_{\text{entry}}) - 0.3 \cdot (d_b - d^{\star}),
\label{eqn:phase1}
\end{equation}
where $V_{\text{entry}} = 0.6 V$, $t_1 = 0.5$\,s is the phase duration, and the back-regulation term prevents the rear gap from collapsing. This phase avoids impulsive acceleration and leverages the predecessor's deceleration (via VP-C) to create a gentle insertion wedge.

\noindent\textbf{Phase 2: Positioning Lock ($0.5 \leq t < 1.5$\,s).} Once the spare has reached cruise speed, it enters a distance-weighted feedback mode to center itself between neighbors. Both front and back gaps are regulated using a shared distance-weighting factor that auto-damps as off-center error grows:
\begin{equation}
w(d_f, d_b) = \frac{d^{\star}}{d^{\star} + \frac{1}{2}|d_f - d_b|},
\label{eqn:weight}
\end{equation}
The velocity update becomes:
\begin{equation}
v_{\text{spare}}^{(2)} = V + 0.2 \cdot w(d_f, d_b) \cdot (d_f - d^{\star}) - 0.5 \cdot w(d_f, d_b) \cdot (d_b - d^{\star}).
\label{eqn:phase2}
\end{equation}
When centered ($d_f \approx d_b \approx d^{\star}$), $w \approx 1$ and gains are nominal. As the spare drifts off-center, $w$ decreases, reducing command magnitude and preventing overshoot. The exponential convergence is achieved through the natural stability of this distance-weighted form.

\noindent\textbf{Phase 3: Transition Approach ($1.5 \leq t \leq 2.0$\,s).} Over the final half-second, the spare progressively increases its front-gap gain from $k_f = 0.2$ to $k_f = 0.5$ (approaching nominal distributed control):
\begin{equation}
k_f^{(3)}(t) = 0.2 + \frac{3}{t_3} \cdot (t - t_2), \quad t_2 = 1.5\text{\,s}, \quad t_3 = 0.5\text{\,s},
\label{eqn:phase3_interp}
\end{equation}
with the velocity update:
\begin{equation}
v_{\text{spare}}^{(3)} = V + k_f^{(3)}(t) (d_f - d^{\star}) - 0.5 \cdot w(d_f, d_b) \cdot (d_b - d^{\star}).
\label{eqn:phase3}
\end{equation}
This smooth transition prevents the spare from suddenly switching to full bidirectional control, reducing the risk of residual oscillation.

\subsubsection{Transition to BASELINE Mode}

After Phase 3 is complete (nominal time $t \geq 2.0$\,s), the spare transitions to BASELINE mode only if all of the following criteria are met:
\begin{enumerate}
    \item \textbf{Time criterion:} $t_{\text{in\_incoming}} \geq 2.0$\,s (ensures all phases execute).
    \item \textbf{Centering criterion:} $|d_f - d_b| < 0.2 \cdot d^{\star}$ (spare is within 20\% of symmetric spacing).
    \item \textbf{Velocity criterion:} $|v_{\text{spare}} - V| < 0.05 \cdot V$ (speed within 5\% of nominal).
    \item \textbf{Gap stability criterion:} $|d_b - d^{\star}| < 0.1 \cdot d^{\star}$ (rear gap stable).
\end{enumerate}

Once all criteria are met, the spare broadcasts $\text{mode} = \textsc{Baseline}$ in its next EPM, neighbors re-enable their back-pressure terms, and the spare operates under standard bidirectional control (Algorithm~\ref{alg:baseline}).

\subsubsection{Rationale and Performance}

The three-phase scheme, enhanced by distance-weighted sensing (Eq.~\ref{eqn:weight}), achieves:
\begin{itemize}
    \item \textbf{Bounded overshoot:} Phase 1 velocity ramp and back-regulation limit spare peak speed to $V + 0.1V$.
    \item \textbf{Fast centering:} Distance-weighted control in Phase 2 provides exponential convergence; typical centering time $\sim 1.0$ to $1.5$\,s.
    \item \textbf{No oscillation:} The auto-damping weight function (Eq.~\ref{eqn:weight}) induces critical damping, preventing sustained oscillations around equilibrium.
    \item \textbf{Low energy overhead:} Typical total energy expenditure during insertion $\approx 7.65$ units (22× reduction vs. uncontrolled entry at 172.6 units).
    \item \textbf{Safety:} Combined with VP-C (which silences predecessor back-pressure), cascading failures are prevented.
\end{itemize}

This stabilization scheme is orthogonal to the domain-space exploration (VP-A, VP-B) and represents an architectural refinement of spare integration that can be validated independently.

\subsection{Validation Strategy}
To ensure credibility and avoid circular validation, we use:
\begin{itemize}
    \item Tooling separation and independent configurations
    \item Analytical bounds for regular geometries
    \item Hardware-in-the-loop (HIL) micro-tests
    \item Cross-validation on hold-out scenarios
    \item Telemetry-based replay and full version/parameter pinning
\end{itemize}
