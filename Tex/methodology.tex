\section{Methodology}
\label{sec:methodology}
 
This section specifies the evaluation protocol used to compare controller variants under identical scenarios. We first state invariant assumptions shared by all runs, then describe the DT/PT simulation--emulation stack, the baseline controller and its variation points (domain space exploration), and finally the parameters and metrics used for comparison.
% Obsolete: legacy overview figure removed (was previously referred to as Fig. 5).
% The updated content is now covered by the UML diagrams (class + state).
\subsection{Methodological Invariants}
All experiments share the following invariant protocols:
\begin{itemize}
 \item\textbf {Code uniformity:}  All experiments enforce CR$^{\circ}$3 by ensuring that every drone executes identical control code.
   \item \textbf{Common control law:} Both PT ('Execute') and DT ('High level simulation') use the same distributed density-balancing algorithm, ensuring predictive consistency.
   
\item \textbf{Architecture coupling:} PT agents execute continuous-time control with discrete sensor sampling; the DT operates in event-driven mode, triggered by exception reporting and outputs spare staging decisions and insertion waypoints.
 
    \item \textbf{Forecast window:} The DT evaluates recovery over an adaptation window $T_{\text{adapt}}$ and applies calibrated thresholds to decide on spare staging.
    \item \textbf{Tooling separation:} To ensure credibility, simulation and emulation use independent configurations and parameter pinning (see Section~\ref{subsec:simulation}).
\end{itemize}

\begin{figure}[t]
    \centering
    % UML diagram is provided as a PDF; keep compilation safe if the file is absent.
    \IfFileExists{figures/UML_DT4RSN.pdf}{%
        \includegraphics[width=\columnwidth]{figures/UML_DT4RSN.pdf}%
    }{%
        \fbox{\parbox[c][0.22\textheight][c]{0.95\columnwidth}{\centering Missing Figure Placeholder: figures/UML_DT4RSN.pdf}}%
    }
    \caption{UML class diagram of the simulation/emulation stack. The base \texttt{Drone} class factors common navigation and control interfaces, while specialized subclasses (\texttt{PT}, \texttt{Process}, \texttt{Avatar}) override \texttt{sense()} and \texttt{execute()} to reflect physical execution, multiprocessing emulation, and pure simulation behaviors. The \texttt{Mission} encapsulates the waypoint/perimeter geometry, and the \texttt{DT} maintains global state and runs event-driven predictive simulations.}
    \label{fig:uml_dt4rsn}
\end{figure}

\begin{figure}[t]
    \centering
    % State diagram for the DT4RSN runtime: nominal deployment/patrolling with detection substates.
    \IfFileExists{figures/UML_DT4RSN_state.pdf}{%
        \includegraphics[width=\columnwidth]{figures/UML_DT4RSN_state.pdf}%
    }{%
        \IfFileExists{figures/UML_DT4RSN_state..pdf}{%
            \includegraphics[width=\columnwidth]{\detokenize{figures/UML_DT4RSN_state..pdf}}%
        }{%
            \fbox{\parbox[c][0.22\textheight][c]{0.95\columnwidth}{\centering Missing Figure Placeholder: figures/UML_DT4RSN_state.pdf}}%
        }%
    }
    \caption{UML state diagram (\texttt{DT4RSN\_STATE}) describing the runtime logic of a drone. After deployment, the drone transitions to a nominal patrolling mode and cycles through waypoint-following. Intrusion handling is modeled as substates of patrolling (e.g., \emph{monitoring}, \emph{confirming}, and \emph{reporting}), and exits back to nominal patrol once the event is cleared. This representation emphasizes that detection augments, rather than replaces, the baseline spacing and navigation loop.}
    \label{fig:uml_dt4rsn_state}
\end{figure}

% NOTE: Any in-text mention of “Fig. 5” for the state diagram is obsolete.
% Always refer to the state diagram via \ref{fig:uml_dt4rsn_state} (currently Fig.~\ref{fig:uml_dt4rsn_state}).

\begin{algorithm}[t]
\caption{Local spacing control with loss reaction and spare assimilation (runs on every drone; no inter-drone comms).}
\label{alg:baseline}
\DontPrintSemicolon
\LinesNumbered
\footnotesize
\SetKwInOut{Input}{Input}
\Input{Nominal speed $V$, max speed $V_{\max}$, desired spacing $d^{\star}$, safety distance $d_{\text{safe}}$, gains $(k_f,k_b,k_{\text{rep}})$, recovery gains $(k_f^{\text{rec}},k_b^{\text{rec}})$, schedule $g(\cdot)$ (default $g\equiv 1$), thresholds $(\alpha,\beta)$, recovery cap $V_{\text{cap}}$.}
\BlankLine
	cp{State (per drone $i$): $mode\in\{\textsc{Baseline},\textsc{Incoming}\}$, $rec\in\{0,1\}$}
\While{mission active}{
    Project pose on perimeter to get curvilinear coordinate $s_i$\;
    Sense front/back neighbors; if missing, set $v_i\leftarrow V$ and continue\;
    Compute ring gaps: $d_f\leftarrow (s_f-s_i)\bmod P$, $d_b\leftarrow (s_i-s_b)\bmod P$\;
    	cp{Recovery trigger (loss hole / compression wave)}
        \label{algline:rec}
    $rec\leftarrow [d_f>\alpha d^{\star}]\ \lor\ [d_b<\beta d^{\star}]$\;
        cc{Symmetric gap gain (front--back difference)}
    $\tilde{k}_{sym} \leftarrow k_{sym}$\;
	\If{$rec$}{ $\tilde{k}_{sym} \leftarrow k_{sym}^{\text{rec}}$\; }
        cc{VP-A: spacing gain schedule (baseline: $g(\cdot)\equiv 1$)}
	$\gamma \leftarrow g\big(|d_f-d^{\star}|\big)$\;
    	\cp{Spacing + safety shaping}
        \label{algline:spacing}
    $v_i\leftarrow V + \tilde{k}_{sym}\,\gamma\,(d_f-d_b)$\;
    \If{$d_f<d_{\text{safe}}$}{ $v_i\leftarrow \min\big(v_i,\ V\cdot(d_f/d_{\text{safe}})\big)$\; }
    \If{$d_b<d_{\text{safe}}$}{ $v_i\leftarrow v_i + k_{\text{rep}}(d_{\text{safe}}-d_b)$\; }
        \label{algline:cap}
    \If{$rec$}{ $v_i\leftarrow \min(v_i, V_{\text{cap}})$\; }
    	cp{Spare: merge until stable, then behave as baseline}
    \If{$mode=\textsc{Incoming}$ \textbf{and} timer expired}{ $mode\leftarrow \textsc{Baseline}$\; }
    $v_i\leftarrow \max\{0,\min(v_i,V_{\max})\}$; move along waypoint loop\;
}
\normalsize
\end{algorithm}

% NOTE: Baseline/variant descriptions are consolidated later in this section
% (see the second "Algorithm Candidates" subsection) to avoid duplication.

\subsection{Perception and intrusion emulation (summary)}
Local awareness is emulated through periodic broadcast of an \emph{Emulated Perception Message} (EPM) by each agent (Avatar in Fig.~\ref{fig:uml_dt4rsn}):
\begin{equation}
epm^t_i = \langle i,\, \mathbf{x}^t_i, \text{mode}_i \rangle
\label{eqn:epm}
\end{equation}
Neighborhood detection is reconstructed from these messages by range filtering. Intruders are represented by negative identifiers and are sensed through the same mechanism with a dedicated monitoring radius. \textit{Critically}, the mode field (BASELINE or INCOMING) is included in the EPM to enable neighbors to adapt their control law in response to spare integration (see VP-C in Section~\ref{subsec:design_space}).

A drone in INCOMING mode (recently inserted spare) broadcasts $\text{mode}_i = \texttt{INCOMING}$ while it is clamped at nominal speed for a fixed hold window (Algorithm~\ref{alg:baseline} lines~82--84). Under the symmetric controller there is no back-pressure term to disable; the flag is only used to gate the release timer and to make the transient explicit in traces.

This mechanism keeps the spare passive during merging without requiring any additional coordination signals beyond the periodic EPM broadcast already used for neighborhood sensing.

\subsection{Simulation Architecture}
\label{subsec:simulation}
The simulation stack is organized into coordinated layers:
\begin{enumerate}
    \item \textbf{DT event-based simulator:} Operates on an event queue, advancing the global state only when an exception is received. Predictive simulation is triggered by events and runs over $T_{\text{adapt}}$.
    \item \textbf{PT software simulation (agent-level):} Each drone is executed as an independent software process running the local control loop (Algorithm~\ref{alg:baseline}). In this simulation mode, ``who is in sight'' can be obtained from the DT (ground-truth neighborhood query) to provide an idealized reference.
    \item \textbf{Perception/messaging emulation (EPM-based):} In the emulation mode, agents have \emph{no} access to DT ground-truth neighborhood information. Instead, sight is reconstructed from sent/received EPMs and their controlled visibility (sampling rate, sensing radius, semantic filtering, exception-only reporting), matching the constrained local view described in Section~V-C.
    \item \textbf{Global supervisor (transversal):} Orchestrates scenarios and hazards (losses, bursts, detections), pins emulator parameters, and performs coherence checks by comparing DT forecasts against PT software-simulation outcomes.
\end{enumerate}
This architecture enables rigorous, scalable, and reproducible evaluation of the system.

% \subsection{Scope note (moved to discussion)}
% DT-enhanced strategies such as loss-to-trajectory prediction and arrival-time-aware insertion are promising extensions, but they are not required to define the experimental protocol of Section~V; we therefore reserve them for the discussion/future-work part of the manuscript.

\subsection{Experimental Parameters and Metrics}
We sweep the following parameters across specified ranges:
\begin{itemize}
    \item \textbf{Fleet configuration:} Fleet size $n$, platform heterogeneity, detection radius $r_d$.
    \item \textbf{Mission geometry:} Polygon complexity $p$, perimeter length $P_{\text{total}}$, waypoint density $w_e$.
    \item \textbf{Fault profiles:} Temporal (instantaneous, sequential, bursts), spatial (uniform, clustered, max-separation).
    \item \textbf{Spare-deployment policy:} Thresholds $\{\sigma^2_{\text{max}}, C_{\text{critical}}, T_{\text{adapt}}, n_{\min}\}$, insertion rule (midpoint of largest gap).
\end{itemize}

Metrics include:
\begin{itemize}
    \item \textbf{Resilience:} Density variance $\sigma^2_{\rho}(t)$, coverage fraction $C(t)$, time-to-restoration $T_{\text{restore}}$, gap statistics $g_{\max}(t)$.
    \item \textbf{Decision:} Spare deployment rate, decision latency $T_{\text{decision}}$, ROC curves, insertion accuracy.
    \item \textbf{Efficiency:} Energy overhead $\Delta E$, PT control cycle time, DT simulation time, scalability.
\end{itemize}
% Methodology D/E: specialized parts that contextualize the above protocol.
% Specialized methodology details (Methodology D/E): design-space exploration and validation.
\subsection{Fulfilling $R_{DT}^\circ$1: Domain Space Exploration}

We operationalize $R_{DT}^\circ$1 by exploring a small, reviewer-auditable design space built around a single baseline controller (Alg.~\ref{alg:baseline}) and two parameter-only variation points. This keeps the controller structure identical across candidates (same sensing, same control loop, same safety shaping) and isolates the effect of changing gains/schedules.

\noindent\textbf{Variation points (explicit replacements).}
\begin{itemize}
    \item \textbf{VP-A (gain scheduling, Alg.~\ref{alg:baseline} line~\ref{algline:spacing}):} replace the baseline spacing update by an error-dependent gain.
    \begin{itemize}
        \item \emph{Baseline (line~\ref{algline:spacing}):}
        $v_i\leftarrow V + k_f(d_f-d^{\star}) - k_b(d_b-d^{\star})$
        \item \emph{Replacement (VP-A):}
        $v_i\leftarrow V + k_f\,g\big(|d_f-d^{\star}|\big)(d_f-d^{\star}) - k_b(d_b-d^{\star})$,
        where $g(\cdot)$ is a nondecreasing schedule with $g(0)=1$.
    \end{itemize}

    \item \textbf{VP-B (recovery gain scheduling, Alg.~\ref{alg:baseline} lines~\ref{algline:rec}--\ref{algline:cap}):} keep the same control structure but switch gains when $rec=1$.
    \begin{itemize}
        \item \emph{Baseline:} use $(k_f,k_b)$ always.
        \item \emph{Replacement (VP-B):} use $(k_f,k_b)\leftarrow (k_f^{\text{rec}},k_b^{\text{rec}})$ while $rec=1$ and revert to nominal gains when $rec=0$.
    \end{itemize}

    \item \textbf{VP-C (adaptive sensing, neighbor state awareness, Alg.~\ref{alg:baseline} line~\ref{algline:spacing}):} disable back-pressure ($k_b$ term) during spare integration to prevent speed collapse.
    \begin{itemize}
        \item \emph{Baseline:}
        $v_i\leftarrow V + k_f\,g\big(|d_f-d^{\star}|\big)(d_f-d^{\star}) - k_b(d_b-d^{\star})$
        \item \emph{Rationale:} When a drone detects that its successor is in INCOMING mode (spare joining), it disables the back-pressure term to allow predecessor to maintain cruise speed rather than decelerating in response to the short transient gap.
        \item \emph{Replacement (VP-C):}
        $v_i\leftarrow V + k_f\,g\big(|d_f-d^{\star}|\big)(d_f-d^{\star}) - k_b(d_b-d^{\star})\cdot\mathbb{1}_{[\text{succ\_not\_incoming}]}$,
        where $\mathbb{1}_{[\text{succ\_not\_incoming}]} = 0$ if successor is in INCOMING mode (detected via local broadcast), else $1$.
    \end{itemize}
\end{itemize}

\subsection{Spare Insertion Stabilization: Three-Phase Control with Distance-Weighted Sensing}

While VP-C prevents the predecessor from crashing during spare insertion, the spare itself requires bounded acceleration to avoid overshoot and extended settling. We address this through a three-phase control scheme where the spare's velocity command is shaped based on its temporal position relative to insertion and its spatial position relative to neighbors.

\subsubsection{Control Phases}

\noindent\textbf{Phase 1: Soft Entry ($0 \leq t < 0.5$\,s).} The spare's initial speed is set to $0.6 \cdot V$ (60\% of fleet nominal speed) to create a smooth relative velocity profile. During this phase, the velocity ramps linearly toward nominal while receiving back-regulation from the rear gap:
\begin{equation}
v_{\text{spare}}^{(1)} = V_{\text{entry}} + \frac{t}{t_1} (V - V_{\text{entry}}) - 0.3 \cdot (d_b - d^{\star}),
\label{eqn:phase1}
\end{equation}
where $V_{\text{entry}} = 0.6 V$, $t_1 = 0.5$\,s is the phase duration, and the back-regulation term prevents the rear gap from collapsing. This phase avoids impulsive acceleration and leverages the predecessor's deceleration (via VP-C) to create a gentle insertion wedge.

\noindent\textbf{Phase 2: Positioning Lock ($0.5 \leq t < 1.5$\,s).} Once the spare has reached cruise speed, it enters a distance-weighted feedback mode to center itself between neighbors. Both front and back gaps are regulated using a shared distance-weighting factor that auto-damps as off-center error grows:
\begin{equation}
w(d_f, d_b) = \frac{d^{\star}}{d^{\star} + \frac{1}{2}|d_f - d_b|},
\label{eqn:weight}
\end{equation}
The velocity update becomes:
\begin{equation}
v_{\text{spare}}^{(2)} = V + 0.2 \cdot w(d_f, d_b) \cdot (d_f - d^{\star}) - 0.5 \cdot w(d_f, d_b) \cdot (d_b - d^{\star}).
\label{eqn:phase2}
\end{equation}
When centered ($d_f \approx d_b \approx d^{\star}$), $w \approx 1$ and gains are nominal. As the spare drifts off-center, $w$ decreases, reducing command magnitude and preventing overshoot. The exponential convergence is achieved through the natural stability of this distance-weighted form.

\noindent\textbf{Phase 3: Transition Approach ($1.5 \leq t \leq 2.0$\,s).} Over the final half-second, the spare progressively increases its front-gap gain from $k_f = 0.2$ to $k_f = 0.5$ (approaching nominal distributed control):
\begin{equation}
k_f^{(3)}(t) = 0.2 + \frac{3}{t_3} \cdot (t - t_2), \quad t_2 = 1.5\text{\,s}, \quad t_3 = 0.5\text{\,s},
\label{eqn:phase3_interp}
\end{equation}
with the velocity update:
\begin{equation}
v_{\text{spare}}^{(3)} = V + k_f^{(3)}(t) (d_f - d^{\star}) - 0.5 \cdot w(d_f, d_b) \cdot (d_b - d^{\star}).
\label{eqn:phase3}
\end{equation}
This smooth transition prevents the spare from suddenly switching to full bidirectional control, reducing the risk of residual oscillation.

\subsubsection{Transition to BASELINE Mode}

After Phase 3 is complete (nominal time $t \geq 2.0$\,s), the spare transitions to BASELINE mode only if all of the following criteria are met:
\begin{enumerate}
    \item \textbf{Time criterion:} $t_{\text{in\_incoming}} \geq 2.0$\,s (ensures all phases execute).
    \item \textbf{Centering criterion:} $|d_f - d_b| < 0.2 \cdot d^{\star}$ (spare is within 20\% of symmetric spacing).
    \item \textbf{Velocity criterion:} $|v_{\text{spare}} - V| < 0.05 \cdot V$ (speed within 5\% of nominal).
    \item \textbf{Gap stability criterion:} $|d_b - d^{\star}| < 0.1 \cdot d^{\star}$ (rear gap stable).
\end{enumerate}

Once all criteria are met, the spare broadcasts $\text{mode} = \textsc{Baseline}$ in its next EPM, neighbors re-enable their back-pressure terms, and the spare operates under standard bidirectional control (Algorithm~\ref{alg:baseline}).

\subsubsection{Rationale and Performance}

The three-phase scheme, enhanced by distance-weighted sensing (Eq.~\ref{eqn:weight}), achieves:
\begin{itemize}
    \item \textbf{Bounded overshoot:} Phase 1 velocity ramp and back-regulation limit spare peak speed to $V + 0.1V$.
    \item \textbf{Fast centering:} Distance-weighted control in Phase 2 provides exponential convergence; typical centering time $\sim 1.0$ to $1.5$\,s.
    \item \textbf{No oscillation:} The auto-damping weight function (Eq.~\ref{eqn:weight}) induces critical damping, preventing sustained oscillations around equilibrium.
    \item \textbf{Low energy overhead:} Typical total energy expenditure during insertion $\approx 7.65$ units (22× reduction vs. uncontrolled entry at 172.6 units).
    \item \textbf{Safety:} Combined with VP-C (which silences predecessor back-pressure), cascading failures are prevented.
\end{itemize}

This stabilization scheme is orthogonal to the domain-space exploration (VP-A, VP-B) and represents an architectural refinement of spare integration that can be validated independently.

\subsection{Validation Strategy}
To ensure credibility and avoid circular validation, we use:
\begin{itemize}
    \item Tooling separation and independent configurations
    \item Analytical bounds for regular geometries
    \item Hardware-in-the-loop (HIL) micro-tests
    \item Cross-validation on hold-out scenarios
    \item Telemetry-based replay and full version/parameter pinning
\end{itemize}
